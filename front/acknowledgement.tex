% !TeX root = ../main.tex

\begin{acknowledgement}

晉侯、秦伯圍鄭,以其無禮於晉,且貳於楚也。晉軍函陵,秦軍氾南。

佚之狐言於鄭伯曰:「國危矣,若使燭之武見秦君,師必退。」公從之。辭曰:「臣之壯也,猶不如人;今老矣,無能為也已。」公曰:「吾不能早用子,今急而求子,是寡人之過也。然鄭亡,子亦有不利焉!」許之,夜縋而出。

見秦伯,曰:「秦、晉圍鄭,鄭既知亡矣。若亡鄭而有益於君,敢以煩執事。越國以鄙遠,君知其難也,焉用亡鄭以陪鄰?鄰之厚,君之薄也。若舍鄭以為東道主,行李之往來,共其乏困,君亦無所害。且君嘗為晉君賜矣,許君焦、瑕,朝濟而夕設版焉,君之所知也。夫晉,何厭之有?既東封鄭,又欲肆其西封,若不闕秦,將焉取之?闕秦以利晉,惟君圖之。」

秦伯說,與鄭人盟。使杞子、逢孫、楊孫戍之,乃還。

子犯請擊之。公曰:「不可。微夫人之力不及此。因人之力而敝之,不仁;失其所與,不知;以亂易整,不武。吾其還也。」亦去之。

\begin{flushright}
左丘明 某月, 魯僖公三十年 謹致於\\
中原
\end{flushright}

\end{acknowledgement}