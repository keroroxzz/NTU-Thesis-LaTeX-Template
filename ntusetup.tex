% !TeX root = ./main.tex

% --------------------------------------------------
% 資訊設定(Information Configs)
% --------------------------------------------------

\ntusetup{
  university*   = {National Taiwan University},
  university    = {國立臺灣大學},
  college       = {電機資訊學院},
  college*      = {College of Electrical Engineering and Computer Science},
  institute     = {電機工程學系},
  institute*    = {Department of Electrical Engineering},
  title         = {中文的標題},
  title*        = {English Title},
  author        = {中文姓名},
  author*       = {英文姓名},
  ID            = {R1234567},
  advisor       = {老師的名字},
  advisor*      = {Lao shih de min zi},
  date          = {2023-07-23},         % 若註解掉,則預設為當天
  oral-date     = {2023-07-23},         % 若註解掉,則預設為當天
  DOI           = {10.6342/NTU202xxxxx},
  keywords      = {範本,測試,大家加油},
  keywords*     = {template, test, every body add oil},
}

% --------------------------------------------------
% 加載套件(Include Packages)
% --------------------------------------------------

\usepackage[numbers]{natbib}            % 參考文獻
\usepackage{amsmath, amsthm, amssymb}   % 數學環境
% \usepackage{ulem, CJKulem}              % 下劃線、雙下劃線與波浪紋效果
\usepackage{booktabs}                   % 改善表格設置
\usepackage{multirow}                   % 合併儲存格
\usepackage{diagbox}                    % 插入表格反斜線
\usepackage{array}                      % 調整表格高度
\usepackage{longtable}                  % 支援跨頁長表格
\usepackage{paralist}                   % 列表環境
\usepackage{algorithm, algpseudocode}   % 演算法
\usepackage{makecell}                   % 窩不知道
\usepackage{tabularx}
\usepackage{tocloft}

\usepackage{lipsum}                     % 英文亂字
\usepackage{zhlipsum}                   % 中文亂字
\usepackage{tikz}
% --------------------------------------------------
% 套件設定(Packages Settings)
% --------------------------------------------------

\DeclareMathOperator*{\argmax}{arg\,max}
\DeclareMathOperator*{\argmin}{arg\,min}
\def\checkmark{\tikz\fill[scale=0.4](0,.35) -- (.25,0) -- (1,.7) -- (.25,.15) -- cycle;} 
